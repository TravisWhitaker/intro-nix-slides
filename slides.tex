\documentclass{beamer}

\usepackage{listings}

% Make Beamer less ugly:
\usefonttheme{serif}
\setbeamercolor{title}{fg=black}
\setbeamercolor{titlelike}{fg=black}
\setbeamercolor{frametitles}{fg=black}
\setbeamertemplate{itemize items}{\color{black}\textbullet}

\frenchspacing

\lstset{basicstyle=\small}
\lstset{language=C}

\begin{document}

\title{Introduction to Nix \\ Purely Functional Package Management}
\author{Travis Whitaker}
\date{}

\maketitle

\begin{frame}
\frametitle{Package Management \textemdash The Stone Age}
\begin{itemize}
\item It's sometime in the late 1980's and you're sitting at a UNIX machine.
\item Your buddy on USENET just sent you an FTP address for some cool new program.
\item Download and extract the tarball on to your machine.
\item Run the Makefile.
\item Copy the binaries to \texttt{/bin}.
\item Read the manual page and write a config file in \texttt{/etc}.
\item Turns out the program was malware. Your machine needs to be serviced by AT\&T.
\end{itemize}
\end{frame}

\begin{frame}
\frametitle{Package Management \textemdash Discovering Fire}
\begin{itemize}
\item It's 1995. Your Sun SPARC workstation runs Debian 0.9.
\item Your buddy on IRC tells you about a cool new XFree86 widget.
\item \texttt{apt-get install some-cool-widget}
\item Aptitude downloads the package from a mirror for you.
\item A flood of text from setup hooks and shell scripts whizzes by.
\item There are new files in \texttt{/bin, /etc, /usr/bin, /usr/lib, /var ...}
\item A few weeks later the widget stops working, so you re-install.
\item \ldots but the setup hooks don't work the same way the second time.
\item A year later most of your packages are broken, so you re-install Debian.
\end{itemize}
\end{frame}

\begin{frame}
\frametitle{Package Management \textemdash Oooh, Shiny!}
\begin{itemize}
\item It's 2013. You're in an Ubuntu VM on a MacBook Pro.
\item Your buddy on Facebook tells you about a cool new tool packaged on NPM.
\item The NPM package actually needs another utility in Ruby, so you grab that Ruby Gem.
\item That Ruby Gem needs some Python tool, so you grab that from PyPI.
\item That Python code needs some Perl code, so you grab that from CPAN.
\item That Perl code has some C dependencies, so you \texttt{apt-get install} those.
\item You wonder if writing Java was really that bad after all\ldots
\end{itemize}
\end{frame}

\begin{frame}
\frametitle{Package Management \textemdash Beginnings of Reproducibility}
\begin{itemize}
\item It's 2016. You work at some hip startup, so everything runs in Docker containers.
\item Your coworker on Slack tells you to try out this new web framework.
\item You download and run the pre-built docker container. Everything \emph{just works}.
\item You want to integrate your slick new app with the latest NoSQL database everyone on HackerNews is talking about.
\item You download the pre-built binary in your Dockerfile.
\item The new database complains about an incompatible \texttt{glibc}.
\item You're not sure what that is, it's probably something only those Linux neckbeards care about.
\item HackerNews has decided that database sucks, so you give up.
\end{itemize}
\end{frame}

\begin{frame}
\frametitle{Development Problems with ``Traditional'' Package Managers}
\begin{itemize}
\item Builds are not reproducible.
\item Imprecisely defined dependencies.
\item Incompatible binaries and libraries can't be used simultaneously.
\item Developer environments are not reproducible; ``it works on my machine!''
\item Packaging and deployment are hard.
\end{itemize}
\end{frame}

\begin{frame}
\frametitle{Deployment Problems with ``Traditional'' Package Managers}
\begin{itemize}
\item Upgrades aren't atomic and can't be easily undone.
\item Adding or upgrading one package can often break others.
\item It's exceedingly difficult to create and maintain identical environments across a fleet.
\item Different programs use wildly different techniques and languages for configuration.
\end{itemize}
\end{frame}

\begin{frame}
\frametitle{How Nix Solves These Problems}
\begin{itemize}
\item One (pure, lazy, functional) language for defining and configuring packages.
\item Each package, its build inputs, and its configuration are cryptographically hashed and immutable.
\item Packages may only reference eachother by these hashes, so dependencies are guaranteed to be tracked precisely.
\item Each package can be safely and transparently cached.
\end{itemize}
\end{frame}

\begin{frame}
\frametitle{The Nix Expression Language}
\begin{itemize}
\item Pure \textemdash no variables, no mutation, etc.
\item Lazy \textemdash only expressions whose results are required are evaluated.
\item Functional \textemdash functions are values.
\item Dynamically typed.
\item JSON-like syntax.
\item Intended to be used for packaging and configuration; not general purpose.
\end{itemize}
\end{frame}

\begin{frame}
\frametitle{Nixpkgs \textemdash Community Provided Package Set}
\begin{itemize}
\item Developed publicly on GitHub.
\item Contains approximately 10000 packages.
\item Built-in machinery for working with lots of existing package managers.
\item Public infrastructure for building, testing, and caching packages.
\end{itemize}
\end{frame}

\begin{frame}
\frametitle{NixOS \textemdash Nix-based Linux Distribution}
\begin{itemize}
\item Developed publicly on GitHub.
\item Uses Nix instead of Aptitude, Pacman, Yum, Portage, etc.
\item All binaries, libraries, and configuration files built by Nix.
\item Package repository provided by Nixpkgs.
\item Atomic upgrades and rollbacks.
\item Supports most desktop environments, window managers, etc.
\item No mutable system state (except for user data, databases, logs, etc.)
\item That means no \texttt{/bin, /etc, /usr, etc.}
\end{itemize}
\end{frame}

\begin{frame}
\begin{center}
\Huge Let's Write a Nix Package!
\end{center}
\end{frame}

\end{document}
